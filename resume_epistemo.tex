\documentclass{report}

\usepackage[utf8]{inputenc}
\usepackage[T1]{fontenc}
\usepackage[francais]{babel}
\usepackage{graphicx}
\graphicspath{{img/}{./}}
\usepackage{hyperref}
\usepackage{textcomp}
\usepackage{amsmath}
\usepackage{geometry}

\usepackage{array,multirow,makecell}
\setcounter{secnumdepth}{3}
\setcounter{tocdepth}{3}

\usepackage{hhline}
\setcellgapes{4pt}
\makegapedcells
\newcolumntype{R}[1]{>{\raggedleft\arraybackslash }b{#1}}
\newcolumntype{L}[1]{>{\raggedright\arraybackslash }b{#1}}
\newcolumntype{C}[1]{>{\centering\arraybackslash }b{#1}}

\usepackage[pages=some]{background}
\backgroundsetup{
	scale=1,
	color=black,
	opacity=0.1,
	angle=0,
	contents={%
		\includegraphics[width=\paperwidth,height=\paperheight]{ulbback.jpg}
	}%
}


\title{\Huge \textbf{\'{E}pistémologie des sciences et des techniques}\\
	\Large TRAN-H-3001\\
	Résumé du cours}

\date{\today}

\author{Mischa MASSON\\
	Contributeurs:\\
	Olivier Hamende
	}



\begin{document}
	
	\begin{figure}[t]
		\includegraphics[width=15cm]{entete.PNG}
	\end{figure}
	
	\maketitle
	
	\renewcommand{\abstractname}{\og Synthèses, Open Source et contributions\fg, Mischa Masson\\ Université Libre de Bruxelles\\2015-2016.}
	
	\BgThispage
	
	\begin{abstract}
		Ce document est une synthèse du cours indiqué dans le titre. Le matériel présenté l'est donc dans le cadre de l'étude du cours et ne peut être repris sans l'accord du professeur concerné.
		
		Veuillez noter que ce document ne remplace en rien l'étude du support originel mais se veut un complément suivant une première étude.
		
		Ce document est ouvert à la modification. Pour ce faire, il vous faudra disposer d'un compte Github et suivre les règles suivantes:
		\begin{itemize}
			\item Les modifications d'orthographe/ syntaxe / grammaire, les corrections de coquilles et autres modifications mineures n'entrainant pas de changement de contenu sont autorisées directement.
			\item Les ajouts de contenu manquant sont à me signaler (via facebook) afin que je puisse les relire et les valider (je suis ouvert et j'accepte d'avoir tord, n'ayez pas peur, le but est d'éviter un texte décousu).
			\item Si vous ne savez pas comment modifier un git, consultez le readme ou contactez moi (via facebook).
			\item Toute personne participant de manière significative au document (relecture complète, corrections/ajouts majeurs,...) sera citée dans les auteurs, merci de me signaler si vous préférez ne pas l'être.
		\end{itemize}
		
		\textbf{Licence Creative Commons}
		Le contenu de ce document est sous la licence Creative Commons :
		Attribution-NonCommercial-ShareAlike 4.0 International (CC BY-NC-SA
		4.0). Celle-ci vous autorise à l’exploiter pleinement, compte- tenu de trois
		choses :
		\begin{enumerate}
			\item Attribution ; si vous utilisez/modifiez ce document vous devez signaler le(s) nom(s) de(s)
		auteur(s).
			\item Non Commercial ; interdiction de tirer un profit commercial de l’œuvre sans autorisation de l’auteur
			\item Share alike ; partage de l’œuvre, avec obligation de rediffuser selon la même licence ou une licence similaire
		\end{enumerate}
		Si vous voulez en savoir plus sur cette licence :\\
		\url{http://creativecommons.org/licenses/by-nc-sa/4.0/}\\
		
		\vspace{12pt}
		\centering\includegraphics[width=.4\textwidth]{CC}
	\end{abstract}
	
	\clearpage
	
	
	\tableofcontents
	
	\chapter{Introduction}
	
	\section{Qu'est-ce que l'épistémologie?}
	
	Du grec "Episteme" et "logos", Connaissance et discours, langage, étude de...; l'épistémologie est donc "l'étude de la connaissance".
	
	On distingue l'anglais \emph{Epistemology}$\rightarrow$étude critique de la connaissance humaine; du francophone \emph{\'{E}pistémologie}$\rightarrow$ étude critique de la connaissance scientifique.
	
	\og L’épistémologie vise fondamentalement à caractériser les sciences en vue de juge de leur valeur (…) \fg	(Soler, 2009)\\
	-Caractériser les sciences? Caractéristiques/critères de démarcation.\\
	-Juger de leur valeur? Quand une théorie scientifique est-elle vraie? Vraie?\\
	Ce sont des questions de philosophie des sciences.
	
	On définit 3 axes de l'épistémo:
	\begin{itemize}
		\item Définir la science.
		\item Juger de la validité d'une théorie scientifique.
		\item Constitution et évolution de la sciences, influences externes (histoire des sciences)
	\end{itemize}
	
	Ce cours concerne l'épistémo des sciences \emph{et techniques}, on définira la nature du savoir technique, les conditions d'apparition et le rapport entre connaissances techniques et scientifiques.
	
	\section{Qu'est-ce que la science?}
	
	\og Ensemble de connaissances, d’études d’une valeur universelle, caractérisées par un objet et une méthode déterminés, et fondées sur des relations objectives vérifiables.\fg (Le Robert)
	
	Il faut définir ce qui est un objet, une méthode, des valeurs universelles, de relations objectives vérifiables,\dots
	
	La science serait donc descriptive et normative, nous y reviendrons.
	
	\section{Techniques et technologie}
	
	Une technique est un art, dérivé en savoir faire. Une technologie est un outil, un objet matériel. La limite est parfois confuse (technologie informatique, par exemple).\\
	On étend technologie à la production des objets, c'est-à-dire leur conception, le prototypage, et la fabrication en soi.
	
	Science vs technique $\Rightarrow$ dicours vs action/activité
	
	Traditionnellement la science comprend le monde et le décrit, elle est neutre et un mauvais usage est imputable a l'utilisateur et non a la science en soi. La technologie répond a un objectif/but en utilisant la science. Cette conception est bouleversée par l'orientation des recherches scientifiques lors des guerres mondiales, par exemple.
	
	\chapter{La démarche scientifique et ses grands concepts}
	
	\section{Concept Aristotélicien}
	
	Attribué a Aristote( IVe s. ACN). Jusqu'à la renaissance (XV-XVIe PCN).
	
	Se base sur la logique, le raisonnement déductif, d'ou une importance du syllogisme (2 phrases "prémisses" sont combinées pour arriver à une conclusion via "le moyen terme" commun aux prémisses). Si les prémisses sont vrais, la conclusion aussi, Aristote utilise le syllogisme pour démontrer des vérités et faire progresser la science.
	
	C'est un enchaînement logique d'idées, pas d'expériences. Il est facile de trouver des syllogismes valides avec une conclusion fausse (Tous les cours sont chouettes. \'{E}pistémo est un cours. \'{E}pistémo est chouette. C'est faux\footnote{Exemple n'engageant aucunement l'avis personnel de l'auteur, cela va de soi.}.).
	
	\section{Aux fondements de la science moderne}
	
	2 penseurs:
	
	\paragraph{Francis Bacon}, avec l'induction. Bacon est Londonien (1561-1626). Il rejette la science des mots d'Aristote et sépare la science concernant les choses des mots en soi. Il accorde de l'importance à l'expérimentation et à la vérification, refuse l'argument d'autorité.
	
	Pour expliquer un phénomène, Aristote privilégie la cause finale càd pour quoi il a lieu\\
	Pour Bacon: La cause mécanique, antérieure au résultat (explication à partir d’une séquence passée) permet de prédire et d’agir; science opératoire et non plus contemplative et verbale.
	
	\og Le savoir est pouvoir.\fg (1597)
	
	La science doit être inductive (dégager des généralités de l'observation du particulier), en opposition au déductif d'Aristote.
	
	\paragraph{René Descartes}, sa raison et sa méthode, un français (1596-1650). Les mathématiques peuvent décrire des phénomènes physiques. Il fondera la méthode scientifique et érige la raison en principe universel.
	
	Il commence par remettre en question tout savoir car rien n'est certain ("Cogito ergo sum", on est sur que de son existence toussa), c'est le Libre-Examen. Ne reste donc que la raison, qu'il explique comment utiliser dans son "discours de la méthode".
	
	4 règles:
	\begin{enumerate}
		\item rejeter l'évidence
		\item analyser les parties
		\item synthétiser
		\item vérifier
	\end{enumerate}
	
	Victoire de l'homme sur la nature par la méthode.
	
	\section{Révolution scientifique}
	
	XVI-XVIIe s. PCN.
	\begin{enumerate}
		\item Nicolas Copernic (Pologne, 1473-1543)
		\item Tycho Brahé (Danemark, 1546-1601)
		\item Johannes Kepler (Allemagne, 1571-1630)
		\item Galilée (Italie, 1564-1642)
		\item Isaac Newton (Angleterre, 1643-1727)
	\end{enumerate}
	
	Aristote, conception géocentriste de l'univers (qui est par ailleurs fini). Présente des irrégularités inexplicables (mvt rétrogrades de planètes,...). Ptolémée (IIe s PCN) explique partiellement avec les épicycles, retour d'un système "stable". Ce système est cependant compliqué et imprécis.
	
	Copernic reprend le système d'Aristarque de Samos (retiens ce nom, padawan), héliocentrique. Le système plus simple lui parait plus plausible mais il doit le compliquer pour rendre compte d'imprécisions et reste incohérent car Copernic garde la cosmologie Aristotélicienne.
	
	Son système ne s'impose pas et est vu comme un outil ne reflétant pas la réalité mais lance le mouvement.
	
	Suit Tycho Brahé  qui refuse l'héliocentrisme mais propose un système solaire (planètes autour du soleil) dont le soleil tourne autour de la Terre. Il invente des outils et trouve des étoiles (incohérence du système d'Aristote qui ne change que sur Terre, l'espace est parfait).
	
	Ouvre la Voie a Johannes Kepler, qui décrit un héliocentrisme avec orbites elliptiques pour mieux décrire les mouvements des planètes. Reste des limites (comètes) et nécessite une nouvelle physique pour "marcher".
	
	Arrive Galilée. Il perfectionne la lunette et l'utilise pour plein de découvertes. Il rendra plausible le système de Copernic. Découvertes notables: satellites de planètes, phases de Vénus, cratères lunaires, astres lointains,\dots
	
	Ceci démontre l'importance de l'instrumentation et de l'observation; et dénote de l'importance de redéfinir la physique.
	
	Expérience du plan incliné, MRU $\Rightarrow$ fin de la physique d'Aristote.
	
	Galilée pratique l'expérimentation, l'induction, prédit les phénomène par les maths, passe du qualitatif au quantitatif\dots
	
	Mais l'église intervient et l'héliocentrisme est décrit comme modèle scientifique ne décrivant pas la réalité.
	
	Enfin, Newton. Synthétise les lois de Kepler, décrit la mécanique classique (gravitation, mouvement). Enfin un lien cohérent entre Kepler et Galilée, la crise initiée par Copernic trouve une fin dans la descritpion mathématique (versus l'interprétation auparavant). Causalité remplace hypothèses.
	
	\chapter{De l'observation à la théorie}
	
	\section{Observation et Expérimentation}
	
	\paragraph{Observation} $\rightarrow$ \og passive \fg. Constat de phénomènes, pas d’intervention dans leur déroulement, mais sélection de ce que l’on observe!
	\paragraph{Expérimentation} $\rightarrow$ active. Recours systématique et rigoureux à l’expérience. Expérience: modification des conditions d’un phénomène et création de situations artificielles (modifications de paramètres).
	
	Expérimenter suppose d'observer, pas l'inverse. (Noter qu'en anglais on sépare "experience" et "experiment", expérience ordinaire et expérience scientifique).
	
	Perspective empiriste: La connaissance ne se fonde que sur l’expérience dont on peut extraire des lois par induction. (Bacon, Newton, Galilée,\dots)
	
	Il est nécessaire pour expérimenter de produire des hypothèses (imagination de l'expérimentateur!) et/ou de constater une relation (objective) entre deux phénomènes.
	
	L'expérience et l'observation requiert des instruments de mesure, qui diffèrent de nos sens par leur approche quantitative et leur précision plus fine.
	
	Il y a donc une perception directe de données brutes ensuite interprétées via des énoncés théoriques, lesquels ont par ailleurs servi a concevoir les instruments. Les instruments de mesure sont des \og théories matérialisées \fg (G. Bachelard). A noter: la précision des instruments disponibles limite celle des théories.
	
	Problèmes:
	\begin{itemize}
		\item Instruments et observations chargés de théorie
		\item On ne peut appréhender que ce que notre cadre conceptuel théorique nous permet de concevoir
		\item Biais d’observation (on ne trouve que ce que l’on cherche)
	\end{itemize}
	
	\section{Qu'est-ce qu'un \og fait\fg scientifique?}
	
	\og Fait\fg vs phénomène
	\paragraph{Phénomène} = ce qui apparaît.
	\paragraph{Fait scientifique} = Ce qui \og se produit \fg + Ce qui est énoncé + Ce sur quoi il y a consensus.
	
	\paragraph{Fait} = \og (…) ce qui peut faire l’objet d’une entente intersubjective au terme d’une vérification, d’une mesure ou d’un contrôle expérimental \fg (Nadeau in Lecourt, 2003, p. 411)
	
	\section{Lois et principes}
	
	Loi induite à partir d’un grand nombre d’observations.\\
	Structure logique d’une loi: Quel que soit x, si x est A, alors, x est B (Sagaut, 2008)
	
	La validité d'une loi est remise en question et mène a des généralisations (Boyle-Mariotte  => Gaz parfaits => Van der Waals). Valable pour les lois quantitatives.
	
	Pour les grands ensembles d'éléments, lois probabilistes.
	
	Une loi n'est donc pas définitive.
	
	Un principe est une loi très générale mais vague (1e loi de Newton= principe d'inertie).
	
	\section{Notion de modèle}
	
	Un modèle est \og un cadre représentatif, idéalisé et ouvert, reconnu approximatif et schématique mais jugé fécond par rapport à un but donné : prévoir, agir sur la nature, la connaître mieux, etc. \fg ((système)Soler, 2009, p. 59)\\
	Un modèle est donc une \og représentation provisoire, dont l’accord avec les théories admises est imparfait (…) \fg (Barberousse (le pirate), 2000, p. 288)
	
	Un modèle est une représentation simplifiée de la réalité, reprenant un nombre fini de caractéristiques jugées pertinentes par la discipline et en fonction de l'objectif. Les fonctions du modèles sont la compréhension, la prévision et l'heuristique (découverte). Les modèles sont plus faciles à utiliser que la théorie pour les problèmes complexes.
	
	On distingue les modèles réduits (maquettes), vivants (rats de labo,...), conceptuels (gaz parfaits,...) et numériques (simulations).
	
	Les modèles ont des avantages de coût, de réduction du danger, éthiques,...
	
	La qualité d'un modèle est fonction de son usage, de savoir s'il est adapté au problème. Un modèle reste un compromis.
	
	\section{Théories scientifiques}
	
	Une théorie scientifique désigne \og un système cohérent qui coordonne, relie et unifie des lois, des hypothèses, des principes et des modèles, les uns apparaissant comme complémentaires des autres \fg (Sagaut, 2008)
	
	Contrairement au modèle qui est plus ou moins simplifié/ajusté, une théorie sera vraie ou fausse. Une théorie non encore validée est une hypothèse (différent du sens d'hypothèse en mathématiques).
	
	Dans les chapitres suivants, retour sur le titre: rôle de l'observation et de l'expérience dans la constitution de théories.
	
	\chapter{La démarche scientifique et ses grands concepts}
	
	\section{L’explication scientifique}
	
	\emph{Ce chapitre est a lire, pas a connaître.}
	
	La science explique des phénomènes. Il existe plusieurs modèles d'explication.
	
	\subsection{L’explication nomologico-déductive (ND)}
	
	Déduction de phénomènes via des lois (nomos) et des faits particuliers.
	
	Utiliser des explanans pour expliquer l'explanandum.
	
	Problèmes:
	\begin{enumerate}
		\item Asymétrie des explications: Modèle ND inapte à rendre compte de l’asymétrie de nombreuses explications
		\item Pertinence: Modèle ND n’impose pas la pertinence comme condition à l’explication scientifique
		\item Nécessité de la référence à une loi Modèle ND impose la présence d’une loi, qui est une exigence trop forte		
	\end{enumerate}
	
	\subsection{L’explication causale-mécaniste}
	
	L’explication doit fournir la cause d’un phénomène $\rightarrow$ Décrire le mécanisme qui produit le phénomène. L’explication causale répond davantage aux critères attendus d’une explication scientifique.
	
	Ce modèle ne s'oppose pas forcément au ND!
	
	\section{Valeur des théories scientifiques}

	Au sens commun, la science dit \og vrai \fg lorsqu’elle adhère aux phénomènes observés $\rightarrow$ Les théories sc. sont \og objectives \fg{}.
	
	Encore faut-il déterminer ce qui est science et ce qui ne l'est pas, prouver le lien théorie-faits et savoir mettre à l'épreuve les théories.
	
	\subsection{La vérité}
	
	\paragraph{La vérité comme correspondance.} Vérité comme correspondance entre un énoncé (ou une théorie) et une réalité extralinguistique: Un énoncé est vrai s’il décrit fidèlement ce qui existe dans la réalité. Conception dominante en Occident depuis l’antiquité.\\
	Critère empirique pour décider si un énoncé est vrai (V) ou faux (F)qui apporte un sentiment de certitude immédiate du à la correspondance entre perception et énoncé d’observation.
	Ne constitue pas une preuve car 2 observateurs ne voient pas nécessairement la même chose $\rightarrow$ La perception ne \og prouve \fg pas, ne \og justifie \fg pas un énoncé d’observation.\\
	Cet énoncé est-il arbitraire pour autant? Non! Il est \og motivé \fg par les perceptions (Karl Popper)\\
	Conclusion: La vérité comme correspondance constitue un objectif à atteindre mais est notion peu opérante car pas de critère empirique permettant de décider du V ou du F d’un énoncé.
	
	\paragraph{La vérité comme consensus.} Un énoncé est vrai si les membres d’une communauté s’accordent à dire qu’il est vrai. Il est vrai parce qu’il y a consensus (et non pas parce qu’il y a correspondance avec la réalité dont découle ce consensus).\\
	Avantage:  Critère empirique opératoire permettant de déterminer si un énoncé est V ou F : y a-t-il consensus ?\\
	Problèmes: Dans la tradition occidentale, la vérité comme correspondance (= vérité absolue) reste l’objectif à atteindre, il est choquant de se dire que la vérité est juste ce sur quoi on est d’accord; car cette vérité est relative et pas infaillible (exemple: géocentrisme = consensus au Moyen Age)\\
	Hypothèse: le consensus n’a pas lieu par hasard, mais est dû à des causes:
	\begin{itemize}
		\item Soit on suppose que le consensus est dû à de “bonnes raisons” (à cause d’une correspondance avec la réalité) $\rightarrow$ V consensus pas trop problématique.
		\item Soit on envisage qu’il puisse être dû à de “mauvaises raisons” (idéologies, personnes charismatiques, …) $\rightarrow$ V consensus n’a plus rien d’absolu.
	\end{itemize}
	Cas d'idéologie sociale: l’affaire Lyssenko (1898-1976). En résumé: opposition a la science bourgeoise $\rightarrow$ pas de concurrence au sein de l'espèce $\rightarrow$ agriculture privilégiée à la recherche $\rightarrow$ exécutions de scientifiques $\rightarrow$ milieux scientifiques communistes occidentaux "contaminés" $\rightarrow$ bordel.\\
	Exemple extrême de l’influence de l’idéologie sur la science.
	
	\paragraph{La vérité comme cohérence.} Un énoncé est vrai s’il s’intègre sans contradiction à un ensemble préalablement admis d’énoncés vrais. Satisfaisant dans les sciences formelles (mathématiques, logique)mais insuffisant dans les sciences empiriques: condition nécessaire mais pas suffisante car aucune référence à la réalité extralinguistique.
	
	\subsection[Lien théorie-objet]{Le problème du lien entre une théorie scientifique et son objet}
	
	\subsubsection{Le statut des entités inobservables?}
	
	Débat entre réalisme et instrumentalisme (=anti-réalisme) à propos des entités inobservables\\
	Pour les entités observables, accord: elles existent et la science vise à les décrire le plus fidèlement possible.\\
	Pour les entités inobservables:
	\begin{description}
		\item[Réalisme] elles existent réellement dans la réalité et la science essaie d’en fournir une description \og vraie \fg (la plus fidèle possible).
		\item[Instrumentalisme] elles ne sont que des fictions introduites par les scientifiques pour prédire des phénomènes observables, mais elles n’existent pas nécessairement dans la réalité.
	\end{description}
	
	\begin{center}
		\begin{tabular}{|p{.5\textwidth}|p{.5\textwidth}|}
			\hline \textbf{Réalisme} & \textbf{Instrumentalisme}\\
			\hhline{|=|=|} Thèse qui affirme \og l’existence [dans la réalité] des entités postulée par les théories scientifiques.\fg  (Tiercelin in Lecourt, p. 802) & Thèse qui affirme que \og nos théories scientifiques ne sont que des moyens calculatoires permettant de prédire des observations \fg (Tiercelin, Ibidem) \\
			\hline Lien: la science parle de choses qui existent dans la réalité extralinguistique. &  Lien : la science parle de choses qui n’ont pas nécessairement d’équivalent dans la réalité. \\
			\hline L’objectif de la science est de fournir une description de la réalité la plus fidèle possible. & L’objectif de la science réside dans sa capacité à prédire des phénomènes. \\
			\hline
		\end{tabular}
	\end{center}
	
	Notion de "vrai":
	\begin{center}
		\begin{tabular}{|p{.5\textwidth}|p{.5\textwidth}|}
			\hline \textbf{Réalisme} & \textbf{Instrumentalisme}\\
			\hhline{|=|=|} Une théorie est vraie si elle correspond (le plus possible) à la réalité & \og ‘vrai’ ne signifie rien de plus que ‘estimé suffisamment établi par une communauté compétente sur base de procédures spécifiables en l’état actuel des connaissances’ \fg (Soler, p. 146) \\
			\hline Vérité comme correspondance (la valeur de vérité d’un énoncé = absolu) &  Vérité comme consensus (la valeur de vérité d’un énoncé = relative) \\
			\hline Pas de critère empirique (car les perceptions altèrent la réalité), induction problématique. Pas d'accès direct à la réalité, arguments indirects (efficacité prédictive,\dots) & Critère empirique en cas de consensus. \\
			\hline
		\end{tabular}
	\end{center}
	
	\subsubsection{Conflit Réalisme/Instrumentalisme}
	
	\paragraph{L’argument de la distinction observable/non-observable.} Est-il possible de tracer une frontière entre observable et non-observable? Si ce n’est pas le cas $\rightarrow$ Argument en faveur du réalisme.
	
	\paragraph{La sous-détermination de la théorie par l’expérience.} Deux théories différentes peuvent être empiriquement équivalentes et les expériences ne suffisent pas à fonder la théorie; donc la théorie précède l’expérience $\rightarrow$ Argument en faveur de l’instrumentalisme.
	
	\paragraph{L’argument du miracle.} Les théories scientifiques permettent de faire des prédictions et de concevoir des technologies “Parce qu’elles sont vraies!”. Si ce n’était pas le cas, ce serait un miracle qu’elles “marchent” si bien $\rightarrow$ Argument en faveur du réalisme. Attention, argument contestable car de nombreuses théories empiriquement efficaces se sont révélées fausses.
	
	\paragraph{L’argument de la corroboration.} Corroboration = confirmation. Argument fort en faveur du réalisme: entité inobservable révélée par différents modes de détection. Proche de l’argument du miracle (coïncidence si l’entité n’existait pas vraiment!). Fortement critiqué (en particulier par Berthelot) car ne correspond pas à la démarche newtonienne de la science: on est au-delà de l’expérience!
	
	Cet argument mène au débat équivalentistes/atomistes, victoire de ces derniers avec Jean Perrin qui calcule le nombre d'Avogadro (et corrobore l'hypothèse des atomes) d'une 12 aine de manières différentes.
	
	\subsection{Le problème de la démarcation}
	
	Définition de critères permettant de distinguer la science de la non-science.
	
	2 critères:
	\begin{itemize}
		\item Le critère de la vérifiabilité
		\item Le critère de la réfutabilité
	\end{itemize}
	
	\paragraph{Vérifiabilité:} défendue par les “vérificationnistes”, entre autres Rudolf Carnap (1891-1970). Tradition empiriste // Bacon, Newton, … La connaissance ne se fonde que sur l’expérience.\\
	Est scientifique ce qui directement ou indirectement vérifiable et est vérifiable ce qui peut être mis en rapport (direct ou indirect) avec les perceptions intersubjectivement validées (\#expérience quoi).
	
	Plusieurs types d’énoncés :
	
	\begin{center}
		\begin{tabular}{|p{.3\textwidth}|p{.3\textwidth}|p{.3\textwidth}|}
			\hline \textbf{Enoncés d’observation} & \textbf{Enoncés théoriques} & \textbf{Enoncés métaphysique}\\
			\hhline{|=|=|=|} concernent les entités publiquement observables & concernent les lois, mais aussi les énoncés qui comportent des termes désignant des entités non observables (PV = cste, objet métallique chargé, etc.) & pas de contenu empirique, donc pas vérifiable et en dehors du champ de la science\\
			\hline sont directement vérifiables par le recours à l’expérience & indirectement vérifiables car pour les vérifier, il faut en déduire des énoncés d’observation qui, eux seront confrontés à l’expérience immédiate & \\
			\hline  & Si tous les énoncés d’observation déduits de l’énoncé théorique sont vérifiés, alors l’énoncé théorique est vrai & \\
			\hline
		\end{tabular}
	\end{center}
	
	Vérifier = réduire à des énoncés d’observation singuliers intersubjectivement validés.
	
	\paragraph{Pour les énoncés d'observation:} un énoncé d’observation est directement vérifiable par l’expérience… Mais problème de la correspondance entre perceptions et énoncés d’observation $\rightarrow$ le cadre théorique influence l’expérience et l’observation des phénomènes. Influence de l'instrumentation, des biais d'observation (on ne trouve que ce qu'on cherche)n du cadre conceptuel,\dots Quels paramètres sont pertinents/à prendre en compte? Influence de la langue sur l'énoncé, de la façon de définir (ie: cygne grand palmipède blanc, mais existent cygnes noirs). Incomplétude de la base empirique en un instant donné.
	
	Conclusion:\\
	Les énoncés d’observation ne sont pas directement vérifiables par l’expérience car:
	\begin{itemize}
		\item la perception motive un énoncé d’observation mais ne le justifie pas
		\item la théorie vient s’immiscer dans les \og faits \fg
		\item le consensus se fait aussi pour des raisons pragmatiques (degré de remaniement nécessaire si l’on accepte un énoncé en conflit avec les théories précédemment acceptées)
	\end{itemize}
	$\Rightarrow$ Difficile de fonder le critère de scientificité sur la vérifiabilité
	
	\paragraph{Pour les énoncés théoriques:} un énoncé théorique est indirectement vérifiable par l’expérience (cas particulier des lois scientifiques universelles dans leur domaine de validité). Point de vue vérificationniste: \og Si tous les énoncés d’observation déduits de l’énoncé théorique sont vérifiés, alors l’énonce théorique est vrai \fg, donc il faudrait vérifier une infinité d’énoncés d’observation pour vérifier une loi!
	
	David Hume (1711-1776): Le problème de l’induction. L’induction ne peut être justifiée de manière rationnelle car fondée sur l’hypothèse de l’uniformité de la nature.\\
	Prouver cette hypothèse? Argument logique? Non! Argument empirique? Non car pour justifier l’induction empiriquement, il faut avoir recours à l’induction $\rightarrow$ raisonnement circulaire $\rightarrow$ L’induction ne peut être justifiée de manière rationnelle, son fondement est psychologique (habitudes) $\rightarrow$ Confiance en l’induction = foi aveugle.
	
	Carnap propose un critère plus faible : Les observations qui servent de prémisses à l’induction ne permettent pas de prouver la vérité d’une conclusion, mais bien de supposer que cette conclusion est (hautement) probable.
	
	Conclusions:\\
	Vérification directe des énoncés d’observation $\rightarrow$ problèmes\\
	Vérification indirecte des énoncés théorique $\rightarrow$ problèmes\\
	 $\Rightarrow$ Pas de sens de prendre le critère de vérifiabilité comme critère de démarcation !
	
	\paragraph{Réfutabilité:} Proposé par Karl Popper (1902- 1994), inspiré de la démarche d'une pierre (Einstein), opposé à la démarche des pseudo-sciences.
	
	Pour rappel, Einstein prétend que la gravité est une courbure de l'espace-temps causant un effet lentille(position apparente d'une étoile derrière un corps célest importante différente de la position réelle).\\
	Il propose de tester son hypothèse allant contre la physique Newtonienne par une "expérience cruciale". Si prédiction fausse, théorie fausse!\\
	Vérifiée en 1919 par Arthur Eddington, 1ère corroboration de la relativité générale.
	
	Démarche scientifique implique une prise de risques selon Popper: il faut tenter de réfuter une hypothèse afin de la valider temporairement, puis tenter de la réfuter encore. Si on ne tente pas de réfuter, on quitte la science.
	
	Est scientifique une théorie qui est réfutable (dont il est possible de démontrer la fausseté. Une théorie réfutable est une théorie qui n’est pas compatible avec n’importe quel résultat expérimental. Pour être scientifique, une théorie doit produire des prédictions définies et précises qui peuvent être testées par l’expérience. Les théories qui ne sont pas réfutables ne sont pas scientifiques, mais métaphysiques, religieuses ou pseudo-scientifiques.
	
	Attention, une théorie peut être scientifique MAIS fausse!
	
	\paragraph{Pseudo-sciences:} non réfutables, compatibles avec n’importe quel résultat expérimental, ne visent pas à se tester, mais à chercher des preuves qui confortent leurs affirmations(\emph{confirmation}, utilisations d'hypothèses ad-hoc en cas de problème).
	
	A noter qu'un énoncé dont on prouve qu'il est faux n'est pas réfutable, seul un énoncé universel est réfutable, un énoncer singulier est vérifiable!
	
	\subsubsection{Statut des maths}
	
	\paragraph{Perspective vérificationniste (Carnap)}
	Enoncés de connaissance? Énoncés directement ou indirectement vérifiables, càd qui peuvent être mis en rapport avec les perceptions publiquement attestables\\
	Énoncés analytiques, prouvés par des principes logiques et des définitions.\\
	Mathématiques $\rightarrow$ énoncés analytiques.
	
	\paragraph{Perspective Poppérienne}
	Démarcation entre sc empiriques et sc formelles: les sc formelles peuvent être déclarées V ou F sur base de leur structure formelle (elles sont déductives) alors que les sc empiriques doivent être testées.\\
	Le critère de réfutabilité ne concerne pas les mathématiques, en tant que sc formelle.
	
	\subsection{Méthode hypothético-déductive}
	
	Popper caractérise la méthode scientifique comme H-D: Formulation d’hypothèses pour en déduire des prédictions qui pourront être testées expérimentalement.
	
	4 étapes:
	\begin{itemize}
		\item Formulation d’hypothèses
		\item Expérimentation (pour tenter de réfuter les hypothèses)
		\item Résultats(interprétation des observations et confrontation aux résultats attendus – si nécessaire, et retour à l’étape 1)
		\item Formulation de lois, de modèles, et de théories
	\end{itemize}
	$\Rightarrow$Tout à fait compatible avec la nécessité de tester les théories, pour tenter de les réfuter.
	
	\subsection{Critique du critère de réfutabilité}
	
	Popper condamne l'utilisation d'hypothèses ad-hoc, cependant celles-ci sont couramment utilisées pour faire avancer la science (découverte de Neptune).
	
	De plus, le critère de Popper est normatif plutôt que descriptif et n'est pas une bonne description de la pratique scientifique.
	
	Existent d'autre critères, comme le scepticisme raisonnable de Jean Bricmon (faire appel au bon sens et aux preuves répétées pour séparer sciences et pseudo-sciences). "Quels arguments me donnez-vous pour qu’il soit plus rationnel de croire ce que vous me dites plutôt que de supposer que vous vous trompez ou que vous me trompez ?"
	
	\section{Notion de preuve}
	
	Conception courante : c’est la preuve qui fait la différence entre connaissance et croyance.
	
	Preuve expérimentale ? Si le résultat d'une expérience valide une hypothèse\\
	Expérience cruciale (EC) ? Francis Bacon, 1620, expérience décisive, qui permettrait, devant deux hypothèses susceptibles d'expliquer un phénomène, d'écarter l'une comme contraire aux faits et de retenir l'autre comme indiscutable. (ex: Einstein et la position réelle, vérifié par Eddington)
	
	Il est parfois compliqué de fournir une preuve expériementale, et parfois on peut prouver plusieurs hypothèses $\Rightarrow$ impossible de déterminer TOUTES les possibilités (ex: lumière onde/corpuscule)
	
	\emph{Même si l’expérience démontre la fausseté d’une hypothèse, rien ne garantit que sa rivale soit vraie, il se peut que ce soit une autre hypothèse qui soit vraie.} Il faudrait énumérer toutes les possibilités (Duhem, 1906).
	
	Le “holisme scientifique” (Duhem et Quine): un scientifique ne peut jamais soumettre au contrôle de l'expérience une hypothèse isolée, mais seulement tout un ensemble d'hypothèses.
	
	$\rightarrow$ Il ne s’agit pas d’abandonner la notion de preuve, mais de prendre conscience de sa complexité et de sa relativité!
	
	Preuve = argumentation plus ou moins rigoureuse en faveur d’une hypothèse. Notion dépend des disciplines (Stengers, 2003). Une preuve est provisoire et pas "unique": La preuve peut résider \og dans une série d’arguments qui se dégagent d’un ensemble de recherches sans qu’il soit possible de dire exactement lequel est vraiment décisif \fg (Lepeltier, 2013)
	
	Principe de précaution: En cas de non certitude scientifique appliquer la es mesures les plus protectives.(version siimplifiée)
	
	
	\chapter[Dynamique des connaissances scientifiques]{La dynamique de la production des connaissances scientifiques}
	
	\section{Facteurs}
	
	\begin{description}
		\item[L’internalisme] ce sont les facteurs internes qui guident l’évolution des sciences. Facteurs internes = facteurs qui interviennent au sein même de l’activité sc. (théories admises, nouvelles hypothèses, procédures de mise à l’épreuve, expériences, démonstrations, concepts disponibles,…)
		\item[L’externalisme] les facteurs externes exercent une influence déterminante dans l’évolution des sciences. Facteurs externes = entre autres contraintes économiques, sociales, religieuses, idéologiques, politiques, etc.
	\end{description}
	
	Entre ces l’internalisme pur et l’externalisme radical : les contraintes internes et les contraintes externes influencent l’évolution des connaissances scientifiques. = Thèse défendue par Gaston Bachelard et Thomas Kuhn (entre autres)
	
	\subsection{Contraintes internes}
	
	\paragraph{Notion d’obstacle épistémologique:}\og ce qui (…) entrave l’accès à une connaissance autre \fg (Soler, 2009). \og (…) c'est dans l'acte même de connaître, intimement, qu'apparaissent, par une sorte de nécessité fonctionnelle, des lenteurs et des troubles \fg (Bachelard, 1938)
	
	Obstacles: habitudes, "symboles inconscients, opinion, préjugés,...
	
	\subsection{Contraintes externes}
	
	Ce n’est vraiment que depuis la seconde guerre mondiale que l’on admet couramment ces influences externes.
	
	Obstacles: politiques, économico-financiers (moi aussi j'invente des mots), sociétaux, institutionnels, idéologiques, religieux,...
	
	\section{Progrès scientifique}
	
	Concept très moderne (apparaît avec Bacon et se développe avec les Lumières). 2 idées centrales : changement et amélioration.
	
	\paragraph{Amélioration:} dépend de la conception de la science (description plus fidèle des phénomènes; meilleure efficacité prédictive; rôle des facteurs esthétiques (voir texte sur la découverte du photon); rôle de l’unité et de la simplicité pour Einstein).
	
	\subsection{Progrès continu}
	
	Conception dominante depuis les lumières:l ’activité scientifique = dévoiler les phénomènes naturels. Un phénomène “dévoilé” le serait une fois pour toutes $\rightarrow$ Processus cumulatif : les nouvelles connaissances viennent s’ajouter aux connaissances antérieures (élargissement du champ des connaissances).
	
	\emph{Conception théologique}
	
	Evolution des sciences guidée par une fin (telos) : le ”dévoilement” des phénomènes naturels
	
	\subsection{Progrès évolutionniste}
	
	Popper (encore): Progrès des théories en direction de la vérité, approximation de plus en plus fine du réel (Popper est réaliste). On apprend des erreurs. La réfutation est la mortalité, les théories adaptées "survivent".\\
	$\Rightarrow$ Pas d'accumulation, progrès discontinu, jamais de certitude absolue.
	
	\subsection[Rupture épistémologique]{Bachelard et la notion de rupture épistémologique}
	
	L’esprit scientifique consiste à poser des problèmes et à tenter de les résoudre. Présence d’obstacles épistémologiques qui entrave l’esprit scientifique. Il faut vaincre ces obstacles.\\
	Le passage de l’ancienne à la nouvelle façon d’appréhender le réel = rupture épistémologique $\rightarrow$ Changements théoriques profonds, qui nécessitent une reconstruction de la raison (Soler(encore) 2009).
	
	Rupture= progrès discontinu (mais pas indépendant des théories précédentes, on s'appuie dessus!)
	
	\subsection[Science normale et révolution]{Kuhn : Science normale et révolutions scientifiques}
	
	Evolution de la science : science normale $\rightarrow$ crise $\rightarrow$ révolution scientifique $\rightarrow$ science normale $\rightarrow$ crise $\rightarrow$ … 
	
	\paragraph{Science normale:} Lorsqu’une discipline est \og mûre \fg $\rightarrow$ régie par un paradigme= ce qui fait consensus au sein d’une communauté scientifique.
	
	\paragraph{Un paradigme comprend :} des contenus théoriques (non remis en question), des méthodes, y compris des techniques et des instruments $\rightarrow$ un savoir-faire pratique, des normes de recherche scientifique
	
	$\Rightarrow$ activité scientifique consiste à préciser les contenus du paradigme, mais pas d’innovation fondamentale. Un paradigme admet un certain nombre de problèmes – ou énigmes – non résolus. Le nombre et la gravité des anomalies sont assez faibles $\rightarrow$ pas de mise en cause des fondements du paradigme.
	
	Lorsque le nombre d’anomalies ou leur gravité augmente, confiance de la communauté scientifique ébranlée $\rightarrow$ la science entre en crise $\rightarrow$ activité scientifique : prolifération de nouvelles théories, potentiellement radicalement différentes des précédents. Cette crise est surmontée par une révolution scientifique qui promeut un nouveau paradigme.
	
	\paragraph{Progrès chez Kuhn:} en science normale, progrès continu. Révolution= progrès discontinu. Le nouveau paradigme est plus efficace (donc meilleur). Au final, progrès discontinu.
	
	\chapter{Démarche de l'ingénieur}
	
	\og Engineering refers to the practice of organizing the design and construction of any artifice which transformsthe physical world around us to meet some recognized need \fg (Rogers, 1983) $\rightarrow$ Conception, production d’artefacts + opérationnalisation (Vincenti, 1990)$\rightarrow$ L’objet de l’ingénierie est la technologie
	
	Avant la révolution industrielle, la technologie évolue indépendamment de la science. La compétition et la concurrence de la révolution impliquent une optimisation des machines, appel aux scientifiques. Science et technologie sont liées.
	
	\section{Spécificité de la connaissance}
	
	Implicitement:\\
	Les connaissances scientifiques $\rightarrow$ Connaissances générées par le scientifique\\
	Les connaissances de l’ingénieur $\rightarrow$ Connaissances utilisées par l’ingénieur.
	
	Couremment le scientifique "dévoile" des phénomènes et l'ingé applique les découvertes.\\
	Ingénierie = science appliquée (pas de conaissances spécifiques)
	
	Cependant, on à vu que la science ne fait pas que dévoiler, et l'ingé développe son propre corpus de connaissances différentes.
	
	\section{Caractérisation de la connaissance de l’ingénieur}
	
	Connaissances liées aux objectifs de l’ingénieur : Concevoir, produire et faire fonctionner un dispositif technique. Inséparabilité des connaissances et de leurs applications pratiques chez l’ingénieur – Savoir orienté.
	
	L'ingé doit faire des compromis pour répondre a des contraintes + importance de l'éthique (tac).\\
	Recherche d'efficacité, acceptabilité de résultats "globaux", exigence de précision, pluridisciplinarité,... Les sciences décrivent les choses telles qu'elles sont, les ingés "telles qu'elles devraient être" (Simon, 1969) pour augmenter le bien-être des individus (Davis, 1995).
	
	L’ingénieur ne se contente donc pas d’appliquer les connaissances scientifiques mais il a aussi recours à d’autres formes de connaissances, à un savoir original (savoir prescriptif, savoir-faire pratique, savoir relatif à la conception,…).
	
	\section{Découverte, invention, innovation}
	
	\paragraph{Découverte:} dévoiler \og un être ou un phénomène qui est déjà là, préexistant \fg (Stengers, 2003). En théorie, la découverte est du ressort de la science. Publiable, mais non appropriable. Publicité essentielle à la découverte scientifique car elle seule permet le contrôle par les pairs qui est l’acte de certification de la découverte (Stengers, 2003). Prcocessus complexe, difficile d'attribuer date et auteur.
	
	\paragraph{Invention:} L’invention \og est le produit de l’esprit ou de l’activité humaine \fg (Stengers, 2003). L’invention porte donc sur quelque chose qui n’existait pas jusqu'alors (>< découverte). Ce qui est créé appartient à l’inventeur, régime de la propriété intellectuelle : l’invention est brevetable. En théorie, l’invention est du ressort de la technologie.
	
	\paragraph{Innovation:} Amélioration d'un produit, d'un service, d'un procédé dans une perspective applicative, afin de produire une plus-value.
	
	\subsection{Problématique des brevets}
	
	Un brevet est un titre de propriété intellectuelle, qui \og vise à rendre publique une invention en empêchant le secret et le monopole, tout en garantissant la rémunération de l’inventeur\fg (Stengers, 2003) $\rightarrow$ Compromis entre la reconnaissance de la propriété intellectuelle et le souci d’utilité sociale.\\
	Les inventions peuvent être brevetées, pas les découvertes, le caractère brevetable se fonde sur la distinction entre naturel et artificiel.
	
	\appendix
	
	\chapter{Textes}
	
	Chaque texte est présenté dans une section. Le titre de la section est celui du texte. On donnera auteur, dates et informations sur le texte, suivis d'une éventuelle note sur l'auteur. Vient ensuite un résumé bref du texte avec les idées principales et une ou deux éventuelles citations.
	
	\section{Exemple}
	Auteur: Mischa Masson; 2016. Mischa est réaliste et voit le progrès comme Khune.
	
	Le texte est un exemple d'une analyse rapide du texte, il montre les points sympa à intégrer.
	
	Passages importants:\\
	\og Bonne chance à tous(\dots)\fg\\
	\og (\dots)merci de votre participation.\fg
	
	\section{Gaston \textsc{{Bachelard}}, \textit{La formation de l'esprit scientifique}, 1934}
	\textit{Thème :} le progrès scientifique discontinu, la neutralité de l'observation. \\
	
	Le progrès de la science se fait avec des obstacles, qui ne sont pas des obstacles externes ou dus à la \og bêtise humaine \fg. Ce sont des obstacles épistémologiques. Il se fait en connaissant \og contre \fg{} la connaissance antérieures, en surmontant ce qui dans l'esprit même fait obstacle à la spiritualisation.
	
	Il est compliqué de se confronter aux seuls faits d'une expérience sans être influencé parce que l'on sait d'auparavant : \og face au réel, ce qu'on croit savoir clairement offusque ce qu'on devrait savoir\fg.
	
	Il faut tout d'abord se défaire de l'opinion et savoir poser les problèmes, de sortes que toute connaissance soit une réponse à une question. Le véritable esprit scientifique, c'est savoir poser les bons problèmes.
	
	À force d'utiliser une idée, celle-ci devient dominante. Le scientifique doit savoir, tout au long de sa vie, se poser des questions pour continuer la croissance spirituelle. Chez certains scientifiques néanmoins, l'instinct formatif persiste face à l'instinct conservatif.
	
	Il s'agit bien d'instincts. Face aux résultats, l'homme sensible garde tous les caractères de sa sensibilité. De plus l'idée scientifique est trop familière, elle se charge d'un concret psychologique trop lourd (métaphores, images, etc.). Même une tête bien faite qui sort de l'Université reste une tête fermé : c'est un produit d'école. 
	
	Lors des révolutions scientifiques, la tête bien faite doit être refaite. L'homme a besoin de muter par les révolutions spirituelles nécessaires à l'invention scientifique.
	
	On répète souvent que la science va vers une unification. Mais en réalité, le progrès scientifique s'est demarqué des arguments philosophique d'unité (unité d'action du Créateur, etc.).
	
	Et quand bien même, il y aurait une certaine unité, l'esprit scientifique serait capable d'aller chercher au delà. En résumé, l'homme animé par l'esprit scientifique désire sans doute savoir, mais c'est aussitôt pour mieux interroger.
	
	\section{René \textsc{Descartes}, \textit{Discours de la méthode}, 1637}
	\textit{Thème :} la méthode scientifique, la valeur de la science. \\
	
	Quand Descartes était jeune, il a du étudier trois sciences assez absconses,  de sorte que, n'y comprenant pas grand chose, il s'est fixé quatre préceptes pour son raisonnement scientifique. Il faut tout d'abord se baser uniquement sur les faits eux-mêmes. Ensuite, analyser les faits indépendamment, en séparant le problème. En faire la synthèse. Enfin, vérifier la théorie par le nombre le plus élevé possible d'expériences différentes.
	
	\section{Galileo \textsc{Galilei}, \textit{Discours concernant deux sciences nouvelles}, 1636}
	Texte traduit par Maurice \textsc{Clavelin} (1995). \textit{Thème : } la validité d'une preuve \\
	
	La proposition énonce qu'en chute libre, $x=gt^2$. Simplicio, adepte de la physique aristotélicienne demande des preuves à Sagredo qui lui explique l'expérience a été maintes et maintes fois reconduite, mais également de manière différente : faire tomber une boule dans un rail de toute sa hauteur (durée de la chute $t_1$), puis à un quart de sa hauteur (durée de la chute $t_2 = \frac{1}{2}\, t_1$), puis en faisant varier l'angle d'inclinaison du rail, etc. Les mesures étaient concordantes, et donc la théorie est corroborée par l'expérience.
	
	\section{Ian \textsc{Hacking}, \textit{Do We See Trough a Microscope?}, 1981}
	To be done
	
	\section{Sve Ove \textsc{Hansson}, \textit{The Epistemology of Technological Risk}, 2005}
	
	
	\paragraph*{Introduction}
	L'analyse de risques est un domaine qui a vu le jour dans les années 1960, à cause d'une certaine inquiétude montante de l'opinion publique face à certaines nouvelles technologies.
	
	Jusqu'à quels points le risque peut-il être compris ? Les connaissances sur le risque sont en réalité des connaissances sur l'inconnu. Ce qui est assez problématique.
	\paragraph*{Le risque en tant que concept quantitatif} Le mot \textit{risque} est utilisé dans différents sens. Le risque est un événement indésirable qui peut (ou pas) arriver. Le risque peut désigner également la cause de cet événement\footnote{La cigarette est un risque pour la santé ; en réalité, le cancer du poumon lié est un risque lié au fait de fumer}, la probabilité pour que cet événement arrive.
	
	En analyse des risques, le risque est défini comme l'impact attendu statistiquement des événements qui peuvent (ou non) avoir lieu. C'est une valeur pondérée par les probabilités que les événements arrivent. Néanmoins, cet impact attendu ne prend pas en compte le contexte sociale des risques\footnote{Par exemple la probabilité $p_1 = \frac{1}{1000}$ de tuer $x_1=1000$ personnes et la probabilité $p_2=\frac{1}{10}$ de tuer $x_2=10$ personnes ont le même impact attend $p_1x_1=p_2x_2=1$ mais correspondent à des réalités tout à fait différentes. Tout comme le fait de savoir \textit{qui} prend le risque.}
	\paragraph*{Le syndrome du smoking}
	Il y a une différence entre risque (probabilités connues ou pouvant être connues) et incertitude (probabilité inconnues, et même dangers inconnus). Dans la vie, il nous faut souvent faire face à des situations où on ne connaît pas la probabilité exacte des dangers qui peuvent nous arriver, voire qu'on ne connaisse pas du tout les dangers qui peuvent nous arriver.
	
	En analyse de risque, où il est nécessaire de quantifier les choses, les personnes en charge de l'analyse de risque souffrent du \emph{syndrome du smoking} : ils prennent des décisions comme s'ils étaient au casino à la roulette et que les dangers et les probabilités étaient connus. Ils pensent être dans la situation de risque et non d'incertitude. Cependant, ce syndrome du smoking peut mener une illusion de contrôle.
	\paragraph*{Probabilités inconnues}
	Il n'est parfois pas possible de déterminer la probabilité d'une certaine panne (par exemple) car l'échantillon de machines ayant subi cette certaine panne n'est pas assez grand. Une manière de palier à ce problème est de créer des chaînes d'événements pouvant conduire à cette certaine panne. Néanmoins, une panne (et un accident en général) peut avoir lieu de plus de manière qu'on ne le croie \emph{a priori}. De plus les différents événements sont généralement dépendants, ce qui complique le calcul de probabilités. Les accidents sont également dus aux erreurs humaines, qui sont pratiquement pas estimables. Et les statistiques sont faites par des humains ce qui influent sur leurs valeurs, comme l'a montré des études psychologiques : on a tendance à être persuadés que nos estimations sont bonnes.
	\paragraph*{Dangers inconnus}
	Il nous faut souvent prendre des décisions sans connaître tous les dangers qui y sont liés. Et heureusement qu'on ne connaît pas tous les dangers liés à nos actions car si on prend en compte les événements de probabilité minime, n'importe laquelle de nos actions peut mener à un désastre. Et il y a des exemples d'expériences qui ont été faites, en ne prenant heureusement pas en compte les dangers lointains qui y étaient liés.
	
	\section{Philippe \textsc{Huneman}, \textit{Le climatoscepticisme relève-t-il de la science ?}, 2015}
	
	\section{Guillaume \textsc{Lecointre}, \textit{Biologie, épistémologie. Des sciences très sollicitées}, 2013}
	
	\section{Nathaniel \textsc{Herzberg}, \textit{La guerre du vide}, 2015}
	
	\section{Pierre \textsc{Marage}, \textit{L'histoire du vide}, 1997}
	
	\section{Anthonie W. M. \textsc{Meijers} et Peter A. \textsc{Kroes}, \textit{Extending the Scope of the Theory of Knowledge}, 2013}
	
	\section{Ali \textsc{Saïb}, \textit{Les virus, inertes ou vivants ?},2006}
	
	\section{Lena \textsc{Soler}, \textit{Anatomie d'une découverte : le photon}, 2001}
	
\end{document}