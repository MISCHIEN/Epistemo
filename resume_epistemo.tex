\documentclass{report}

\usepackage[utf8]{inputenc}
\usepackage[T1]{fontenc}
\usepackage[francais]{babel}
\usepackage{graphicx}
\graphicspath{{img/}{./}}
\usepackage{hyperref}
\usepackage{textcomp}
\usepackage{amsmath}
\usepackage{geometry}

\usepackage{array,multirow,makecell}
\setcounter{secnumdepth}{3}
\setcounter{tocdepth}{3}

\usepackage{hhline}
\setcellgapes{4pt}
\makegapedcells
\newcolumntype{R}[1]{>{\raggedleft\arraybackslash }b{#1}}
\newcolumntype{L}[1]{>{\raggedright\arraybackslash }b{#1}}
\newcolumntype{C}[1]{>{\centering\arraybackslash }b{#1}}

\usepackage[pages=some]{background}
\backgroundsetup{
	scale=1,
	color=black,
	opacity=0.1,
	angle=0,
	contents={%
		\includegraphics[width=\paperwidth,height=\paperheight]{ulbback.jpg}
	}%
}


\title{\Huge \textbf{\'{E}pistémologie des sciences et des techniques}\\
	\Large TRAN-H-3001\\
	Résumé du cours}

\date{\today}

\author{Mischa MASSON\\
	Contributeurs:\\
	nope
	}



\begin{document}
	
	\begin{figure}[t]
		\includegraphics[width=15cm]{entete.PNG}
	\end{figure}
	
	\maketitle
	
	\renewcommand{\abstractname}{\og Synthèses, Open Source et contributions\fg, Mischa Masson\\ Université Libre de Bruxelles\\2015-2016.}
	
	\BgThispage
	
	\begin{abstract}
		Ce document est une synthèse du cours indiqué dans le titre. Le matériel présenté l'est donc dans le cadre de l'étude du cours et ne peut être repris sans l'accord du professeur concerné.
		
		Veuillez noter que ce document ne remplace en rien l'étude du support originel mais se veut un complément suivant une première étude.
		
		Ce document est ouvert à la modification. Pour ce faire, il vous faudra disposer d'un compte Github et suivre les règles suivantes:
		\begin{itemize}
			\item Les modifications d'orthographe/ syntaxe / grammaire, les corrections de coquilles et autres modifications mineures n'entrainant pas de changement de contenu sont autorisées directement.
			\item Les ajouts de contenu manquant sont à me signaler (via facebook) afin que je puisse les relire et les valider (je suis ouvert et j'accepte d'avoir tord, n'ayez pas peur, le but est d'éviter un texte décousu).
			\item Si vous ne savez pas comment modifier un git, consultez le readme ou contactez moi (via facebook).
			\item Toute personne participant de manière significative au document (relecture complète, corrections/ajouts majeurs,...) sera citée dans les auteurs, merci de me signaler si vous préférez ne pas l'être.
		\end{itemize}
		
		\textbf{Licence Creative Commons}
		Le contenu de ce document est sous la licence Creative Commons :
		Attribution-NonCommercial-ShareAlike 4.0 International (CC BY-NC-SA
		4.0). Celle-ci vous autorise à l’exploiter pleinement, compte- tenu de trois
		choses :
		\begin{enumerate}
			\item Attribution ; si vous utilisez/modifiez ce document vous devez signaler le(s) nom(s) de(s)
		auteur(s).
			\item Non Commercial ; interdiction de tirer un profit commercial de l’œuvre sans autorisation de l’auteur
			\item Share alike ; partage de l’œuvre, avec obligation de rediffuser selon la même licence ou une licence similaire
		\end{enumerate}
		Si vous voulez en savoir plus sur cette licence :\\
		\url{http://creativecommons.org/licenses/by-nc-sa/4.0/}\\
		
		\vspace{12pt}
		\centering\includegraphics[width=.4\textwidth]{CC}
	\end{abstract}
	
	\clearpage
	
	
	\tableofcontents
	
	\chapter{Introduction}
	
	\section{Qu'est-ce que l'épistémologie?}
	
	Du grec "Episteme" et "logos", Connaissance et discours, langage, étude de...; l'épistémologie est donc "l'étude de la connaissance".
	
	On distingue l'anglais \emph{Epistemology}$\rightarrow$étude critique de la connaissance humaine; du francophone \emph{\'{E}pistémologie}$\rightarrow$ étude critique de la connaissance scientifique.
	
	\og L’épistémologie vise fondamentalement à caractériser les sciences en vue de juge de leur valeur (…) \fg	(Soler, 2009)\\
	-Caractériser les sciences? Caractéristiques/critères de démarcation.\\
	-Juger de leur valeur? Quand une théorie scientifique est-elle vraie? Vraie?\\
	Ce sont des questions de philosophie des sciences.
	
	On définit 3 axes de l'épistémo:
	\begin{itemize}
		\item Définir la science.
		\item Juger de la validité d'une théorie scientifique.
		\item Constitution et évolution de la sciences, influences externes (histoire des sciences)
	\end{itemize}
	
	Ce cours concerne l'épistémo des sciences \emph{et techniques}, on définira la nature du savoir technique, les conditions d'apparition et le rapport entre connaissances techniques et scientifiques.
	
	\section{Qu'est-ce que la science?}
	
	\og Ensemble de connaissances, d’études d’une valeur universelle, caractérisées par un objet et une méthode déterminés, et fondées sur des relations objectives vérifiables.\fg (Le Robert)
	
	Il faut définir ce qui est un objet, une méthode, des valeurs universelles, de relations objectives vérifiables,\dots
	
	La science serait donc descriptive et normative, nous y reviendrons.
	
	\section{Techniques et technologie}
	
	Une technique est un art, dérivé en savoir faire. Une technologie est un outil, un objet matériel. La limite est parfois confuse (technologie informatique, par exemple).\\
	On étend technologie à la production des objets, c'est-à-dire leur conception, le prototypage, et la fabrication en soi.
	
	Science vs technique $\Rightarrow$ dicours vs action/activité
	
	Traditionnellement la science comprend le monde et le décrit, elle est neutre et un mauvais usage est imputable a l'utilisateur et non a la science en soi. La technologie répond a un objectif/but en utilisant la science. Cette conception est bouleversée par l'orientation des recherches scientifiques lors des guerres mondiales, par exemple.
	
	\chapter{La démarche scientifique et ses grands concepts}
	
	\section{Concept Aristotélicien}
	
	Attribué a Aristote( IVe s. ACN). Jusqu'à la renaissance (XV-XVIe PCN).
	
	Se base sur la logique, le raisonnement déductif, d'ou une importance du syllogisme (2 phrases "prémisses" sont combinées pour arriver à une conclusion via "le moyen terme" commun aux prémisses). Si les prémisses sont vrais, la conclusion aussi, Aristote utilise le syllogisme pour démontrer des vérités et faire progresser la science.
	
	C'est un enchaînement logique d'idées, pas d'expériences. Il est facile de trouver des syllogismes valides avec une conclusion fausse (Tous les cours sont chouettes. \'{E}pistémo est un cours. \'{E}pistémo est chouette. C'est faux\footnote{Exemple n'engageant aucunement l'avis personnel de l'auteur, cela va de soi.}.).
	
	\section{Aux fondements de la science moderne}
	
	2 penseurs:
	
	\paragraph{Francis Bacon}, avec l'induction. Bacon est Londonien (1561-1626). Il rejette la science des mots d'Aristote et sépare la science concernant les choses des mots en soi. Il accorde de l'importance à l'expérimentation et à la vérification, refuse l'argument d'autorité.
	
	Pour expliquer un phénomène, Aristote privilégie la cause finale càd pour quoi il a lieu\\
	Pour Bacon: La cause mécanique, antérieure au résultat (explication à partir d’une séquence passée) permet de prédire et d’agir; science opératoire et non plus contemplative et verbale.
	
	\og Le savoir est pouvoir.\fg (1597)
	
	La science doit être inductive (dégager des généralités de l'observation du particulier), en opposition au déductif d'Aristote.
	
	\paragraph{René Descartes}, sa raison et sa méthode, un français (1596-1650). Les mathématiques peuvent décrire des phénomènes physiques. Il fondera la méthode scientifique et érige la raison en principe universel.
	
	Il commence par remettre en question tout savoir car rien n'est certain ("Cogito ergo sum", on est sur que de son existence toussa), c'est le Libre-Examen. Ne reste donc que la raison, qu'il explique comment utiliser dans son "discours de la méthode".
	
	4 règles:
	\begin{enumerate}
		\item rejeter l'évidence
		\item analyser les parties
		\item synthétiser
		\item vérifier
	\end{enumerate}
	
	Victoire de l'homme sur la nature par la méthode.
	
	\section{Révolution scientifique}
	
	XVI-XVIIe s. PCN.
	\begin{enumerate}
		\item Nicolas Copernic (Pologne, 1473-1543)
		\item Tycho Brahé (Danemark, 1546-1601)
		\item Johannes Kepler (Allemagne, 1571-1630)
		\item Galilée (Italie, 1564-1642)
		\item Isaac Newton (Angleterre, 1643-1727)
	\end{enumerate}
	
	Aristote, conception géocentriste de l'univers (qui est par ailleurs fini). Présente des irrégularités inexplicables (mvt rétrogrades de planètes,...). Ptolémée (IIe s PCN) explique partiellement avec les épicycles, retour d'un système "stable". Ce système est cependant compliqué et imprécis.
	
	Copernic reprend le système d'Aristarque de Samos (retiens ce nom, padawan), héliocentrique. Le système plus simple lui parait plus plausible mais il doit le compliquer pour rendre compte d'imprécisions et reste incohérent car Copernic garde la cosmologie Aristotélicienne.
	
	Son système ne s'impose pas et est vu comme un outil ne reflétant pas la réalité mais lance le mouvement.
	
	Suit Tycho Brahé  qui refuse l'héliocentrisme mais propose un système solaire (planètes autour du soleil) dont le soleil tourne autour de la Terre. Il invente des outils et trouve des étoiles (incohérence du système d'Aristote qui ne change que sur Terre, l'espace est parfait).
	
	Ouvre la Voie a Johannes Kepler, qui décrit un héliocentrisme avec orbites elliptiques pour mieux décrire les mouvements des planètes. Reste des limites (comètes) et nécessite une nouvelle physique pour "marcher".
	
	Arrive Galilée. Il perfectionne la lunette et l'utilise pour plein de découvertes. Il rendra plausible le système de Copernic. Découvertes notables: satellites de planètes, phases de Vénus, cratères lunaires, astres lointains,\dots
	
	Ceci démontre l'importance de l'instrumentation et de l'observation; et dénote de l'importance de redéfinir la physique.
	
	Expérience du plan incliné, MRU $\Rightarrow$ fin de la physique d'Aristote.
	
	Galilée pratique l'expérimentation, l'induction, prédit les phénomène par les maths, passe du qualitatif au quantitatif\dots
	
	Mais l'église intervient et l'héliocentrisme est décrit comme modèle scientifique ne décrivant pas la réalité.
	
	Enfin, Newton. Synthétise les lois de Kepler, décrit la mécanique classique (gravitation, mouvement). Enfin un lien cohérent entre Kepler et Galilée, la crise initiée par Copernic trouve une fin dans la descritpion mathématique (versus l'interprétation auparavant). Causalité remplace hypothèses.
	
	\chapter{De l'observation à la théorie}
	
	\section{Observation et Expérimentation}
	
	\paragraph{Observation} $\rightarrow$ \og passive \fg. Constat de phénomènes, pas d’intervention dans leur déroulement, mais sélection de ce que l’on observe!
	\paragraph{Expérimentation} $\rightarrow$ active. Recours systématique et rigoureux à l’expérience. Expérience: modification des conditions d’un phénomène et création de situations artificielles (modifications de paramètres).
	
	Expérimenter suppose d'observer, pas l'inverse. (Noter qu'en anglais on sépare "experience" et "experiment", expérience ordinaire et expérience scientifique).
	
	Perspective empiriste: La connaissance ne se fonde que sur l’expérience dont on peut extraire des lois par induction. (Bacon, Newton, Galilée,\dots)
	
	Il est nécessaire pour expérimenter de produire des hypothèses (imagination de l'expérimentateur!) et/ou de constater une relation (objective) entre deux phénomènes.
	
	L'expérience et l'observation requiert des instruments de mesure, qui diffèrent de nos sens par leur approche quantitative et leur précision plus fine.
	
	Il y a donc une perception directe de données brutes ensuite interprétées via des énoncés théoriques, lesquels ont par ailleurs servi a concevoir les instruments. Les instruments de mesure sont des \og théories matérialisées \fg (G. Bachelard). A noter: la précision des instruments disponibles limite celle des théories.
	
	Problèmes:
	\begin{itemize}
		\item Instruments et observations chargés de théorie
		\item On ne peut appréhender que ce que notre cadre conceptuel théorique nous permet de concevoir
		\item Biais d’observation (on ne trouve que ce que l’on cherche)
	\end{itemize}
	
	\section{Qu'est-ce qu'un \og fait\fg scientifique?}
	
	\og Fait\fg vs phénomène
	\paragraph{Phénomène} = ce qui apparaît.
	\paragraph{Fait scientifique} = Ce qui \og se produit \fg + Ce qui est énoncé + Ce sur quoi il y a consensus.
	
	\paragraph{Fait} = \og (…) ce qui peut faire l’objet d’une entente intersubjective au terme d’une vérification, d’une mesure ou d’un contrôle expérimental \fg (Nadeau in Lecourt, 2003, p. 411)
	
	\section{Lois et principes}
	
	Loi induite à partir d’un grand nombre d’observations.\\
	Structure logique d’une loi: Quel que soit x, si x est A, alors, x est B (Sagaut, 2008)
	
	La validité d'une loi est remise en question et mène a des généralisations (Boyle-Mariotte  => Gaz parfaits => Van der Waals). Valable pour les lois quantitatives.
	
	Pour les grands ensembles d'éléments, lois probabilistes.
	
	Une loi n'est donc pas définitive.
	
	Un principe est une loi très générale mais vague (1e loi de Newton= principe d'inertie).
	
	\section{Notion de modèle}
	
	Un modèle est \og un cadre représentatif, idéalisé et ouvert, reconnu approximatif et schématique mais jugé fécond par rapport à un but donné : prévoir, agir sur la nature, la connaître mieux, etc. \fg ((système)Soler, 2009, p. 59)\\
	Un modèle est donc une \og représentation provisoire, dont l’accord avec les théories admises est imparfait (…) \fg (Barberousse (le pirate), 2000, p. 288)
	
	Un modèle est une représentation simplifiée de la réalité, reprenant un nombre fini de caractéristiques jugées pertinentes par la discipline et en fonction de l'objectif. Les fonctions du modèles sont la compréhension, la prévision et l'heuristique (découverte). Les modèles sont plus faciles à utiliser que la théorie pour les problèmes complexes.
	
	On distingue les modèles réduits (maquettes), vivants (rats de labo,...), conceptuels (gaz parfaits,...) et numériques (simulations).
	
	Les modèles ont des avantages de coût, de réduction du danger, éthiques,...
	
	La qualité d'un modèle est fonction de son usage, de savoir s'il est adapté au problème. Un modèle reste un compromis.
	
	\section{Théories scientifiques}
	
	Une théorie scientifique désigne \og un système cohérent qui coordonne, relie et unifie des lois, des hypothèses, des principes et des modèles, les uns apparaissant comme complémentaires des autres \fg (Sagaut, 2008)
	
	Contrairement au modèle qui est plus ou moins simplifié/ajusté, une théorie sera vraie ou fausse. Une théorie non encore validée est une hypothèse (différent du sens d'hypothèse en mathématiques).
	
	Dans les chapitres suivants, retour sur le titre: rôle de l'observation et de l'expérience dans la constitution de théories.
	
	\chapter{La démarche scientifique et ses grands concepts}
	
	\section{L’explication scientifique}
	
	\emph{Ce chapitre est a lire, pas a connaître.}
	
	La science explique des phénomènes. Il existe plusieurs modèles d'explication.
	
	\subsection{L’explication nomologico-déductive (ND)}
	
	Déduction de phénomènes via des lois (nomos) et des faits particuliers.
	
	Utiliser des explanans pour expliquer l'explanandum.
	
	Problèmes:
	\begin{enumerate}
		\item Asymétrie des explications: Modèle ND inapte à rendre compte de l’asymétrie de nombreuses explications
		\item Pertinence: Modèle ND n’impose pas la pertinence comme condition à l’explication scientifique
		\item Nécessité de la référence à une loi Modèle ND impose la présence d’une loi, qui est une exigence trop forte		
	\end{enumerate}
	
	\subsection{L’explication causale-mécaniste}
	
	L’explication doit fournir la cause d’un phénomène $\rightarrow$ Décrire le mécanisme qui produit le phénomène. L’explication causale répond davantage aux critères attendus d’une explication scientifique.
	
	Ce modèle ne s'oppose pas forcément au ND!
	
	\section{Valeur des théories scientifiques}

	Au sens commun, la science dit \og vrai \fg lorsqu’elle adhère aux phénomènes observés $\rightarrow$ Les théories sc. sont \og objectives \fg{}.
	
	Encore faut-il déterminer ce qui est science et ce qui ne l'est pas, prouver le lien théorie-faits et savoir mettre à l'épreuve les théories.
	
	\subsection{La vérité}
	
	\paragraph{La vérité comme correspondance.} Vérité comme correspondance entre un énoncé (ou une théorie) et une réalité extralinguistique: Un énoncé est vrai s’il décrit fidèlement ce qui existe dans la réalité. Conception dominante en Occident depuis l’antiquité.\\
	Critère empirique pour décider si un énoncé est vrai (V) ou faux (F)qui apporte un sentiment de certitude immédiate du à la correspondance entre perception et énoncé d’observation.
	Ne constitue pas une preuve car 2 observateurs ne voient pas nécessairement la même chose $\rightarrow$ La perception ne \og prouve \fg pas, ne \og justifie \fg pas un énoncé d’observation.\\
	Cet énoncé est-il arbitraire pour autant? Non! Il est \og motivé \fg par les perceptions (Karl Popper)\\
	Conclusion: La vérité comme correspondance constitue un objectif à atteindre mais est notion peu opérante car pas de critère empirique permettant de décider du V ou du F d’un énoncé.
	
	\paragraph{La vérité comme consensus.} Un énoncé est vrai si les membres d’une communauté s’accordent à dire qu’il est vrai. Il est vrai parce qu’il y a consensus (et non pas parce qu’il y a correspondance avec la réalité dont découle ce consensus).\\
	Avantage:  Critère empirique opératoire permettant de déterminer si un énoncé est V ou F : y a-t-il consensus ?\\
	Problèmes: Dans la tradition occidentale, la vérité comme correspondance (= vérité absolue) reste l’objectif à atteindre, il est choquant de se dire que la vérité est juste ce sur quoi on est d’accord; car cette vérité est relative et pas infaillible (exemple: géocentrisme = consensus au Moyen Age)\\
	Hypothèse: le consensus n’a pas lieu par hasard, mais est dû à des causes:
	\begin{itemize}
		\item Soit on suppose que le consensus est dû à de “bonnes raisons” (à cause d’une correspondance avec la réalité) $\rightarrow$ V consensus pas trop problématique.
		\item Soit on envisage qu’il puisse être dû à de “mauvaises raisons” (idéologies, personnes charismatiques, …) $\rightarrow$ V consensus n’a plus rien d’absolu.
	\end{itemize}
	Cas d'idéologie sociale: l’affaire Lyssenko (1898-1976). En résumé: opposition a la science bourgeoise $\rightarrow$ pas de concurrence au sein de l'espèce $\rightarrow$ agriculture privilégiée à la recherche $\rightarrow$ exécutions de scientifiques $\rightarrow$ milieux scientifiques communistes occidentaux "contaminés" $\rightarrow$ bordel.\\
	Exemple extrême de l’influence de l’idéologie sur la science.
	
	\paragraph{La vérité comme cohérence.} Un énoncé est vrai s’il s’intègre sans contradiction à un ensemble préalablement admis d’énoncés vrais. Satisfaisant dans les sciences formelles (mathématiques, logique)mais insuffisant dans les sciences empiriques: condition nécessaire mais pas suffisante car aucune référence à la réalité extralinguistique.
	
	\section[Lien théorie-objet]{Le problème du lien entre une théorie scientifique et son objet}
	
	\subsection{Le statut des entités inobservables?}
	
	Débat entre réalisme et instrumentalisme (=anti-réalisme) à propos des entités inobservables\\
	Pour les entités observables, accord: elles existent et la science vise à les décrire le plus fidèlement possible.\\
	Pour les entités inobservables:
	\begin{description}
		\item[Réalisme] elles existent réellement dans la réalité et la science essaie d’en fournir une description \og vraie \fg (la plus fidèle possible).
		\item[Instrumentalisme] elles ne sont que des fictions introduites par les scientifiques pour prédire des phénomènes observables, mais elles n’existent pas nécessairement dans la réalité.
	\end{description}
	
	\begin{center}
		\begin{tabular}{|p{.5\textwidth}|p{.5\textwidth}|}
			\hline \textbf{Réalisme} & \textbf{Instrumentalisme}\\
			\hhline{|=|=|} Thèse qui affirme \og l’existence [dans la réalité] des entités postulée par les théories scientifiques.\fg  (Tiercelin in Lecourt, p. 802) & Thèse qui affirme que \og nos théories scientifiques ne sont que des moyens calculatoires permettant de prédire des observations \fg (Tiercelin, Ibidem) \\
			\hline Lien: la science parle de choses qui existent dans la réalité extralinguistique. &  Lien : la science parle de choses qui n’ont pas nécessairement d’équivalent dans la réalité. \\
			\hline L’objectif de la science est de fournir une description de la réalité la plus fidèle possible. & L’objectif de la science réside dans sa capacité à prédire des phénomènes. \\
			\hline
		\end{tabular}
	\end{center}
	
	Notion de "vrai":
	\begin{center}
		\begin{tabular}{|p{.5\textwidth}|p{.5\textwidth}|}
			\hline \textbf{Réalisme} & \textbf{Instrumentalisme}\\
			\hhline{|=|=|} Une théorie est vraie si elle correspond (le plus possible) à la réalité & \og ‘vrai’ ne signifie rien de plus que ‘estimé suffisamment établi par une communauté compétente sur base de procédures spécifiables en l’état actuel des connaissances’ \fg (Soler, p. 146) \\
			\hline Vérité comme correspondance (la valeur de vérité d’un énoncé = absolu) &  Vérité comme consensus (la valeur de vérité d’un énoncé = relative) \\
			\hline Pas de critère empirique (car les perceptions altèrent la réalité), induction problématique. Pas d'accès direct à la réalité, arguments indirects (efficacité prédictive,\dots) & Critère empirique en cas de consensus. \\
			\hline
		\end{tabular}
	\end{center}
	
	\subsection{Conflit Réalisme/Instrumentalisme}
	
	\paragraph{L’argument de la distinction observable/non-observable.} Est-il possible de tracer une frontière entre observable et non-observable? Si ce n’est pas le cas $\rightarrow$ Argument en faveur du réalisme.
	
	\paragraph{La sous-détermination de la théorie par l’expérience.} Deux théories différentes peuvent être empiriquement équivalentes et les expériences ne suffisent pas à fonder la théorie; donc la théorie précède l’expérience $\rightarrow$ Argument en faveur de l’instrumentalisme.
	
	\paragraph{L’argument du miracle.} Les théories scientifiques permettent de faire des prédictions et de concevoir des technologies “Parce qu’elles sont vraies!”. Si ce n’était pas le cas, ce serait un miracle qu’elles “marchent” si bien $\rightarrow$ Argument en faveur du réalisme. Attention, argument contestable car de nombreuses théories empiriquement efficaces se sont révélées fausses.
	
	\paragraph{L’argument de la corroboration.} Corroboration = confirmation. Argument fort en faveur du réalisme: entité inobservable révélée par différents modes de détection. Proche de l’argument du miracle (coïncidence si l’entité n’existait pas vraiment!). Fortement critiqué (en particulier par Berthelot) car ne correspond pas à la démarche newtonienne de la science: on est au-delà de l’expérience!
	
	Cet argument mène au débat équivalentistes/atomistes, victoire de ces derniers avec Jean Perrin qui calcule le nombre d'Avogadro (et corrobore l'hypothèse des atomes) d'une 12aine de manières différentes.
	
	
	
\end{document}